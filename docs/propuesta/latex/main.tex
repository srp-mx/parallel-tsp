\documentclass{article}
\usepackage{../../util/estilo}

\renewcommand{\TituloMain}[0]{Propuesta de Proyecto}
\newcommand{\SubtituloMain}[0]{Implementación de un Algoritmo Genético Paralelo para el Problema del Agente Viajero}
\newcommand{\MateriaMain}[0]{Complejidad Computacional}
\newcommand{\FechaMain}[0]{07 de Mayo de 2025}

\nocite{*}

\begin{document}
    \begin{titlepage}
    \begin{centering}

        \thispagestyle{empty}
        
        \setlength{\parindent}{0cm}
        
        \rule{\linewidth}{0.1mm}
        \begin{center}
            \begin{minipage}{2.5cm}
                \begin{center}
                \includegraphics[height=2.7cm]{Logo_UNAM.png}
                \end{center}
            \end{minipage}\hfill
            %--------------
            \begin{minipage}{10cm}
                \begin{center}
                \textbf{ Universidad Nacional Autónoma de México}\\[0.1cm]
                \textbf{Facultad de Ciencias}\\[0.1cm]
                \textbf{\MateriaMain}\\[0.1cm]
                \textbf{Semestre 2025-2}\\[0.1cm]
                \textbf{Integrantes: }\\[0.1cm]
                319526240 - Romero Palacios Santiago\\[0.1cm]
                320268135 - Nava Córdova Mariana\\[0.1cm]
                \FechaMain
                \end{center}
            \end{minipage}\hfill
            %--------------
            \begin{minipage}{2.5cm}
                \begin{center}
                \includegraphics[height=2.7cm]{Logo_FC.png}
                \end{center}
            \end{minipage}
        \end{center}
        \rule{\linewidth}{0.1mm}
        
        
        
        \date{}
        \vspace{0.5cm}
        \vspace{0.5cm}
            {\scshape\LARGE \textcolor{Gray}{\TituloMain} \par}
            {\scshape\Huge \textcolor{RoyalBlue}{\SubtituloMain} \par}
            {
                \vspace*{2mm}
                %\begin{center}\begin{minipage}{6.5cm}
                %\centering \tableofcontents
                %\end{minipage}\end{center}
            }
        \vspace{0.5cm}

        \subsection*{Descripción}

        El proyecto consistirá en elaborar y evaluar un programa que resuelva
        el problema del agente viajero mediante un algoritmo genético en
        paralelo, tanto en CPU como en GPU. Para ello, se realizará una
        implementación en secuencial y dos en paralelo: en CPU con
        \href{https://www.openmp.org/}{OpenMP} y GPU con
        \href{https://developer.nvidia.com/cuda-toolkit}{CUDA}. El lenguaje de
        elección en todo caso será C++. Nos interesa conocer qué tan
        eficientemente podemos obtener buenas aproximaciones de problemas
        NP-Duros con hardware moderno de grado de consumidor, usando el
        problema del agente viajero euclidiano en su versión de optimización
        como caso de estudio. En GPU seguiremos cercanamente a \cite{ipn_gpu},
        y para la versión secuencial a \cite{pav_tesis}. Utilizaremos las
        recomendaciones de evaluación de metaheurísticas en paralelo de
        \cite{alba_ch2}.
        
        \vspace{1cm}
        \vfill
    \end{centering}
    \printbibliography
    \end{titlepage}

    \pagestyle{logotipos}

\end{document}

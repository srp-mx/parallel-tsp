\documentclass[main.tex]{subfiles}
\usepackage{util/estilo}

\begin{document}

La logística mueve a la economía mundial, siendo un factor crítico para el éxito
o fracaso de muchos negocios que hacen uso de la importación y exportación de
bienes. \parencite{logistica}

Las pequeñas y medianas empresas (PYMES) a menudo se enfrentan a limitaciones
de recursos que hacen que las soluciones logísticas tradicionales y de alto
costo sean poco viables. Esto dificulta el escalamiento y crecimiento de sus
operaciones, pues rápidamente se vuelven altamente ineficientes por la
intractabilidad de obtener soluciones buenas manualmente.
\parencite{logistica}\parencite{logistica2}

Existen alternativas de renta de tiempo de cómputo, pero esto puede no ser
conveniente si se trabaja con información sensible, además de generar una
dependencia en el proveedor del servicio en las operaciones de la organización.
\parencite{logistica2} Aprovechar hardware de grado de consumidor relativamente
económico para resolver problemas logísticos computacionalmente difíciles es
una alternativa práctica y rentable.

Nos interesa investigar cómo aprovechar al máximo los recursos computacionales
de CPU y GPU económicos en la implementación de algoritmos de optimización para
generar mejoras significativas en el rendimiento sin necesidad de hardware más
especializado ni depender de la renta de recursos de cómputo. Usamos el
Problema del Agente Viajero Euclidiano como un caso de estudio de juguete.

\end{document}

\documentclass[main.tex]{subfiles}
\usepackage{util/estilo}

\begin{document}

\begin{cajaEnunciado}
    \addcontentsline{toc}{subsubsection}{Problema del Agente Viajero}
    \subsubsection*{Problema del Agente Viajero}
\end{cajaEnunciado}

El problema del agente viajero (PAV o bien TSP) consiste en encontrar el camino
más corto que permita a un viajero recorrer un conjunto de ciudades exactamente
una vez y regresar al punto de origen. Dado que pertenece a la clase de
problemas NP-duro (también conocido como NP-difícil), su resolución exacta se
vuelve intractable conforme el número de ciudades crece, lo que hace
imprescindible el uso de heurísticas y metaheurísticas para obtener soluciones
aproximadas de calidad. \parencite{pav_tesis} \parencite{cormen}

La tesis de \cite{pav_tesis} aborda diferentes enfoques heurísticos, como el
método del vecino más cercano, inserción aleatorizada y la mejora mediante
búsqueda local con 2-intercambio (2-opt). Estas estrategias se conocen como
metaheurísticas de trayectoria, pues se sigue un camino en búsqueda del óptimo.
Otra técnica que aborda, es la del algoritmo genético, que exploraremos a
continuación.

\begin{cajaEnunciado}
    \addcontentsline{toc}{subsubsection}{Algoritmos Genéticos}
    \subsubsection*{Algoritmos Genéticos}
\end{cajaEnunciado}

Los algoritmos genéticos son una clase de algoritmos evolutivos inspirados en
la teoría de la evolución de Darwin \parencite{genintro}. Operan sobre una
población de soluciones codificadas como cromosomas (en el problema del agente
viajero, permutaciones de ciudades) y aplican operadores de selección,
cruzamiento y mutación para generar nuevas soluciones.

Su fuerte es su capacidad para explorar amplias regiones del espacio de
búsqueda y evitar caer fácilmente en óptimos locales. Algunos enfoques
recientes han combinado estos algoritmos con técnicas como “extinciones
masivas” \parencite{tsp_ext} o restricciones propias de problemas relacionados
como el de rutas de vehículos (VRP) \parencite{prv_gen}.

\begin{cajaEnunciado}
    \addcontentsline{toc}{subsubsection}{Algoritmos Paralelos}
    \subsubsection*{Algoritmos Paralelos}
\end{cajaEnunciado}

Dado que muchos de los pasos en algoritmos genéticos, como la evaluación de
aptitud, la selección o la cruza pueden ejecutarse de manera independiente, su
implementación en paralelo suele mejorar considerablemente la eficiencia,
especialmente cuando se trata de instancias de gran tamaño.
\parencite{paralevol} \parencite{alba_ch2}

Para implementar este paralelismo, se han desarrollado herramientas como OpenMP
y CUDA, ampliamente utilizadas en el cómputo de alto rendimiento.

OpenMP es una interfaz de programación que facilita la ejecución paralela en
arquitecturas de memoria compartida, como los procesadores multinúcleo.
Esto nos permite dividir automáticamente tareas repetitivas como los bucles en
distintos hilos de ejecución. \parencite{omp}

Por otro lado, CUDA es una plataforma desarrollada por NVIDIA que permite
ejecutar código en GPU, las cuales están diseñadas para realizar miles de
operaciones en paralelo con sus numerosos CUDA cores (así como otro tipo de
núcleos de procesamiento), agrupados en Streaming Multiprocessors, que a su vez
se agrupan en TPCs y estos en GPCs. Esto convierte a CUDA en una herramienta
práctica para acelerar cálculos intensivos, como la evaluación masiva de rutas
o la aplicación simultánea de operadores genéticos. \parencite{cuda}
\parencite{ipn_gpu}

\end{document}
